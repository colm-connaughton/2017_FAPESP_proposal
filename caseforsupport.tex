%a. A substantive description of the exchange activities, emphasizing their relevance. The proposal must state clearly how the exchange activities to be carried out will contribute to the ongoing research project funded by FAPESP and to the research being carried by the partner researcher abroad;
%
%b. A timeline for each specific exchange mission, considering the limit of missions and funding established in this Call;
%
%c. Performance indicators for the planned activities;
%
%d. A description of each candidate?s contribution to the mission, explaining their expertise to carry out the foreseen activities;
%
%e. Foreseen actions that will add to the impact of the exchange for the Partner Institution and for the Host Institution in the State of S�o Paulo, e.g. by means of seminars, short courses and visits to other institutions that carry out research activities in cognate areas;

\section{Summary of proposed exchange activities}
A partnership and scientific exchange programme is proposed between the University of S�o Paulo (USP) and the University of Warwick (UoW). The USP partner is the Institute of Mathematics and Computer Science (ICMC). The UoW partners are the Centre for Complexity Science (CCS) and the Zeeman Institute for Systems Biology and Infectious Disease Epidemiology Research (SBIDER), two  interdisciplinary joint ventures of the Warwick Mathematics Institute. The USP project leader is Prof. Francisco A. Rodrigues, an associate professor and head of the Complex Systems group at the ICMC. The UoW project leader is Dr. Colm Connaughton, a reader in applied mathematics and director of the CCS. The scientific theme  of the proposed partnership is dynamical processes on multi-layer and dynamic networks. It relates to the FAPESP grant "Information spreading in complex networks" (2016/25682-5) currently held by Prof. Rodrigues at the ICMC. The high level aims of the proposed exchange are:
\begin{itemize}
\item to improve scientific understanding of the dynamics of coherent phenomena in dynamical and multilayer networks.
\item to forge a new, durable and mutually beneficial scientific collaboration between existing strong complex systems research groups at UoW and USP.
\end{itemize}
A detailed description of the scientific objectives of the collaboration is provided in Sec.~\ref{sec-science}. The proposed exchange programme will run for two years spanning the 2017-18 and 2018-19 academic years. The first year will see visits to UoW by members of the USP team and the second year will see return visits to USP by members of the UoW team. A detailed description of the proposed schedule is provided in Sec.~\ref{sec-timeline}. In parallel to this proposal, Connaughton and Rodrigues have jointly prepared a complementary proposal to be submitted to the Leverhulme Trust (deadline 11 May 2017) seeking a Leverhulme Visiting Professorship to be held by Prof. Rodrigues at UoW for the 2017-18 academic year. If funded, this complementary proposal will allow us to significantly enhance both the breadth and depth of the proposed partnership.

\section{Description of research team}

\subsection{Partner institutions}
The {\hvnb Centre for Complexity Science (CCS)} at the UoW was founded in 2007 to host one of the first EPSRC Centres for Doctoral Training (CDT) on the topic of complex systems. It grew rapidly to become one of the largest complexity science research groups in the UK. It currently hosts 8 academic staff, several postdoctoral researchers and about 50 graduate students. It co-hosts the \pounds 3.4M EPSRC-MRC Centre for Doctoral Training in Mathematics of Real World Systems in partnership with SBIDER. The {\hvnb Zeeman Institute for Systems Biology and Infectious Disease Epidemiology Research (SBIDER)} was established in 2016 when the Warwick Systems Biology Centre and Warwick Infectious Disease and Epidemiology Research Centre joined forces to form one of the largest mathematical biology research groups in the UK. In addition to partnering with the CCS in the MathSys CDT it is the coordinating node of a \$10M global consortium funded by the Gates Foundation to develop the modelling capability to support the UN 2020 goals for the elimination of Neglected Tropical Diseases. Both the CCS and SBIDER are interdisciplinary joint ventures of the {\hvnb Warwick Mathematics Institute (WMI)} which provides physical space and administrative support. WMI ranked 3rd in the UK for mathematical sciences in the 2014 Research Excellence Framework (the most recent UK national research assessment exercise) and 32nd globally in the 2016 QS world university subject rankings. In 2014, Regius Professor Martin Hairer was awarded a fields medal. UoW is therefore ideally placed to participate in and benefit from international partnerships with distinguished researchers like Prof. Rodrigues. {\hvnb  Institute of Mathematics and Computer Science (ICMC), USP}\\
{\color{red} Need a few lines from Francisco about ICMC}

\subsection{University of S\~{a}o Paolo Team}
\begin{itemize}
\item {\hvnb Prof. Francisco A. Rodrigues (USP team leader)} is an associate professor at the Institute of Mathematics and Computer Science (ICMC), University of S\~{a}o Paulo. He holds a degree in Physics (2001, B.A. degree) and a masters degree in Computational Physics (2004), both from University of S\~{a}o Paulo. In 2007, he got his PhD in Physics from Physics Institute of S\~{a}o Carlos (University of S\~{a}o Paulo). After that, he was a post-doctoral researcher at the same institute with FAPESP fellowship (2007-2010). In the beginning of 2010, he got a tenure-track position at the ICMC as a lecturer and researcher. He became associate professor on the end of 2013. Francisco has published more than 60 papers in scientific journals, receiving more than 2,500 citations (Google scholar). He concluded the supervision of three masters students and advised several undergraduate students. Currently, he supervises five PhD students and six masters students. His teaching activities in graduate and undergraduate courses are related to the fields of Statistics, Probability, Stochastic Processes and Network Science. He also coordinated several research projects (including international projects).
Currently, he is the president of the research committee at the ICMC and the head of the Complex Systems group.

\item {\hvnb Dr. Thomas Kau\^{e} Dal Maso Peron}
{\color{red} Need a few lines from Francisco for Thomas' track record}
\item {\hvnb Mr. Guilherme Ferraz de Arruda (PhD student)}
{\color{red} Need a line or two from Francisco about Guilherme's PhD research}
\item {\hvnb Mr. Paulo C�sar Ventura (PhD student)}
{\color{red} Need a line or two from Francisco about Paolo's PhD research}
\end{itemize}

\subsection{University of Warwick Team}
\begin{itemize}
\item
{\hvnb Dr. Colm Connaughton (UoW team leader)}  is a Reader in the Warwick Mathematics Institute, the director of the Warwick Centre for Complexity Science and the deputy director of the EPSRC-MRC funded Mathematical of Real World Systems (MathSys) Centre for Doctoral Training. He is an expert on complex systems, fluid dynamics and non-equilibrium statistical mechanics with particular expertise on turbulence and the kinetic theory of irreversible systems. He has published 51 papers in international peer-reviewed journals and is responsible, as PI or CI, for over \pounds 11M of current or previous EPSRC CDT support, including the MathSys CDT, plus over \pounds 350K of direct research support from various agencies. 
\item {\hvnb Prof. Matt Keeling:} is a Professor of mathematical epidemiology at the University of Warwick with a joint appointment between the Mathematics Institute and the School of Life Sciences. His research focuses on the development and application of mathematical models of infectious disease propagation in human and animal populations.  He has authored over 100 papers which have garnered over 11000 citations (Google Scholar). He is currently the director of the Zeeman Institute for Systems Biology and Infectious Disease Epidemiology (SBIDER).
\item {\hvnb Prof. Robert MacKay FRS FInstP FIMA:} is the Professor of Interdisciplinary Mathematics at the Warwick Mathematics Institute, a fellow of the Royal Society and former president of the Institute of Mathematics and its Applications. He is the former director of the Centre for Complexity Science and founding director of the MathSys CDT. He is an expert in dynamical systems, mathematical physics and complexity science. He is particularly well known for his fundamental contribution to the mathematical theory of synchronisation and coherent structures in spatially extended nonlinear systems.   He has authored 1 book, 136 peer-reviewed journal articles and 50 other articles, and entered the ISI Highly cited list under Mathematics in 2008.
\item{\hvnb Dr. Samuel Johnson} is an assistant professor in the WMI and CCS.  He is an expert in networks and theoretical ecology.
\item{\hvnb Dr. Louise Dyson} is an assistant professor in the WMI and SBIDER.  Her expertise is in social complexity and mathematical epidemiology.
\item{\hvnb Dr. Emre Esenturk} is a Marie Curie postdoctoral fellow in the WMI and CCS. He is an expert in kinetic theory.
\item{\hvnb Dr. Lorenzo Pellis} is a Henry Dale postdoctoral fellow in the WMI and SBIDER. His field of expertise is mathematical epidemiology.
\end{itemize}

\section{Scientific objectives}
\label{sec-science}
\subsection{Background and scientific context}
Complex systems are composed of connected elements whose interactions are nonlinear~\cite{BarYam97,Mitchell09}. Examples of complex systems include our society, the Internet, our brain and cellular interactions~\cite{Mitchell09}. The modelling of complex system behaviour has attracted the attention of researchers from several areas, including Mathematics, Physics, Biology, Computer Science and Engineering. Indeed, the study of these systems has several applications in Science and Technology, including the control of disease spreading~\cite{Pastor015} and adjustment of synchronization in data transmission~\cite{Pikovsky03}. Since the last century, researchers have verified that the structure of several complex systems are characterized by a very heterogeneous organization, presenting a special set of highly connected elements, whereas the remainder of the components are low connected~\cite{Barabasi99}. This ubiquitous scale-free architecture is observed in numerous systems, from food webs to collaboration networks of scientists~\cite{Costa011}, and has important implication to system dynamics. For instance, previous works verified that the epidemic threshold for disease propagation depends on the level of network heterogeneity, being close to zero in a scale-free network~\cite{Pastor015}. Thus, these networks are the ideal medium for disease transmission, explaining the quick propagation of viruses in our society, like those that can cause influenza~\cite{Keeling08}. Moreover, the network organization influences the emergence of synchronization in systems like power grids and our brain~\cite{Arenas08}. In this case, a set of self-sustained oscillators interacts and after a given critical coupling strength, a collective behaviour emerges. Other dynamical processes, including cascade failures, percolation and voter models, also depend on the network structure~\cite{Barrat08}. 

Although the study of dynamical processes on complex networks provided important results to understand the structure and behaviour of complex systems, most of these works considered static networks composed by only one type of interaction~\cite{Costa011, Barabasi016}. However, most of natural and artificial systems evolve with time, changing contacts due to numerous factors. For instance, people move from home to workplaces everyday and their contacts change throughout the day. Moreover, groups of animals move to form complex dynamical patterns across a broad range of length scales, like flock of birds. These systems are better modelled as dynamical agents moving in a surface, defining networks whose structure changes with time~\cite{Buscarino08, Levis017}.  Additionally, real world systems present more than one type of interaction. For instance, in cities, people can move on the streets or by subways, defining different means of transport.  Multilayer networks can represent these multiple channels of connections, where each layer defines one type of interaction~\cite{Kivela014, Boccaletti014}. 

The study of complex systems by moving agents~\cite{Levis017} and multilayer networks~\cite{Kivela014} is very recent and there are several issues to be addressed, as we propose in this project. Particularly, we aim at developing a new model of synchronization of traveling agents, considering different types of interaction and rough landscapes for particles moving. We will be able to determine conditions for phase transition and the influence of parameters related to the dynamics of agents on the emergence of the collective behaviour. Epidemic and rumour spreading will also be analysed on the context of traveling particles to determine how different properties, such as velocity and roughness of the landscape, influence the propagation. Regarding multilayer networks, we will analyse how the propagation of awareness in a layer affects the disease transmission in the other layer. With this study, we will be able to develop methods for epidemic control based on dissemination of rumours about the pathogen. The research proposed in this project will enable to start the collaboration between the Centre for Complexity Science at the University of Warwick and the Complex Systems Group at the University of S�o Paulo. This collaboration will include several researchers and graduate students from both sides.

{\color{red} Need to add other members of USP team to these objectives based on expertise}

\subsection{Objective 1:  Synchronisation of moving agents (Rodrigues, MacKay, Connaughton, Johnson, Esenturk)}
Synchronization occurs when a set of self-sustained oscillators are weakly connected as a network, whose structure influences the emergence of the synchronous state. In the case of Kuramoto oscillators, the critical coupling strength for the emergence of synchronization depends inversely on the second moment of the degree distribution. Chaotic oscillators, including those proposed by R�ssler and Lorentz, can also synchronize when linked as a network~\cite{Arenas08}. Most of the studies on synchronization of coupled oscillators have considered static networks. However, recently, researchers considered oscillators interacting in a dynamic network. In this case, oscillators move in a two-dimensional space and interact inside a fixed radius. The initial works on this topic considered oscillators as interacting mobile agents~\cite{Levis017} in a two-dimensional space. Our goal is to generalize this study by considering non-flat surfaces. That is, we aim at implementing a model of particles moving in a rough relief whose velocities change according to their positions. Each agent is a Kuramoto oscillator and the interaction between pairs of oscillators occurs according to their distance. This analysis will enable to verify how such terrain affects the emergence of the synchronous state. This model generalizes previous works and combines the experience of prof. Francisco with synchronization and the expertise of prof. Colm with mass transport models. This work will also be developed with collaboration of prof. Robert Mackay, who works with rigorous mathematical formalism, which may be necessary for a theoretical formulation of the system dynamics. Prof. Samuel Johnson will also participate on this analysis, since he has experience on complex networks theory and modelling of complex systems.

\subsection{Objective 2: Epidemic spreading in complex networks (Rodrigues, Keeling, Dyson, Pellis)}

Epidemic processes are ubiquitous in nature, society and technology~\cite{Pastor015}. The theoretical study of information propagation has several applications, from disease control and forecasting to rumour spreading analysis~\cite{Barrat08, Pastor015}. Prof. Francisco has been working with epidemic models in networks for five years (e.g.~\cite{Arruda014, Arruda017}) and the collaboration with researchers from Warwick, mainly professors Lorenzo Pellis, Louise Dyson and Matt Keeling, will complement several studies on computational epidemiology. The main idea is to develop epidemic models on systems composed of moving agents or represented as multilayer networks. In the former case, subjects travel in a two-dimensional space and interact when their distance is smaller than a given threshold. This modelling approach was considered before~\cite{Frasca06, Buscarino08}, but the initial models are very simple and do not include important aspects of epidemic dynamics, such as heterogeneous transmission and agents moving with different velocities. Our goal is to simulate several epidemic models, like susceptible-infected-recovered (SIR), on different scenarios of agent mobility. Theoretical analysis to obtain analytical solutions of the dynamical processes, such as the epidemic threshold, will also be addressed during the visit of prof. Francisco to Warwick. In the case of multilayer representation, we will consider networks composed by two types of interactions. In one layer, information about a disease is shared among users, like in a social network. This information affects the transmission of the disease on the other layer, where the pathogen travels. Our goal is to generalize previous works on this topic~\cite{Funk09, Granell013} and use more suitable models for information transmission, like the one introduced by Maki and Thompson~\cite{Moreno04}. Thus, we will be able to verify how epidemics can be controlled by the propagation of awareness. 
These works will be developed in collaboration between prof. Francisco Rodrigues and prof. Matt Keeling, Lorenzo Pellis and Louise Dyson, who study epidemic models. 


\subsection{Objective 3: Network methods for analysis of turbulence (Rodrigues, Connaughton)}
Time series analysis is a central topic in several research areas, including physics, biology, medicine and economics. Recently, several methods have been developed to transform time series into networks (e.g.~\cite{Lacasa08, Marwan09, Donner010}). This approach has been used to find patterns of activity in dynamical systems. For instance, the use of recurrence network analysis to model paleoclimatic data enabled to identify transition periods related to human evolution~\cite{Donges011}. 

There are several methods to transform time series into networks~\cite{Lacasa08, Marwan09} and there is no consensus about what is the most appropriate one for a given application. In case of turbulence, it is possible to extract time series whose patterns determines different regimes. Our goal is to compare methods to transform time series into complex networks and verify which one is the most suitable to find turbulent regimes. In addition, we aim at determining how properties of turbulent flow are related to network topology. That is, we will verify how flow regimes in fluid dynamics are associated to network topology. For instance, whether the increase in network heterogeneity is related to chaotic behaviour. This study will provide a new approach to study turbulence dynamics and open a new area of research. This research will be performed in collaboration between prof. Colm Connaughton and prof. Francisco Rodrigues.

\subsection{Relationship to ongoing FAPESP grant}
{\color{red} Need to write something sensible here}
Fapesp grant: Information spreading in complex networks
Process: 2016/25682-5

Epidemic spreading can follow different routes, which can be mapped as layers in social networks. A great challenge in computational epidemics todays is to determine how the structure of the network influences the propagation. In this project, we aim at applying statistical inference methods for quantification of which network properties most influence the propagation of disease and rumors. In addition, we will model the social organization as made up by several layers. Finally, we will study how awareness influences the propagation of an infectious agent. These studies will contribute for developing new methods for epidemic forecasting and control.

\section{Management, timeline and and expected outputs}
\label{sec-timeline}
\subsection{Management and scientific schedule}
The project will be managed jointly by Connaughton and Rodrigues following the rules and procedures of their respective institutions. All the projects propose here will be developed in parallel, since we have researchers interested in different fields. As following, we provide a description of the activities and the respective expected period.

\begin{itemize}
\item {\hvnb Synchronization of moving agents:} Initially, the project will start with some simulations of synchronization in a two dimensional flat space. The group from USP will perform the software implementation with help of researchers from Warwick. It is expected that this step will take about three months and the first results will be discussed after this time period. It is expected the whole analysis of this part of the project will take about 12 months.

\item {\hvnb Network methods for analysis of turbulence:} This part of the project will take about 18 months. We will implement several algorithms to transform time series into networks. After that, we will validate the implementations considering time series of several dynamical processes, including chaotic, random and periodic signals. Next, the group from Warwick will generate data related to turbulence. Then, we will compare the methods in order to verify which one is the most suitable to capture properties of turbulent flow.

\item {\hvnb Epidemic spreading in complex networks:} The analysis of epidemic processes in networks will take about 18 months. Initially, we will adapt the models developed by the Warwick group to multilayer and temporal networks. Next, we will generalize disease transmission models to temporal networks made up of moving agents. Both groups will perform the implementation of the models.
\end{itemize}

\subsection{Schedule of exchange visits}

To develop the research proposed in this project, several meetings are necessary to discuss the results and further research steps. Thus, basically, we plan to have the following visits:
\begin{itemize}
\item From Warwick to USP:
\begin{itemize}
\item Prof. Colm  and Emre Esenturk will visit the group in S�o Carlos in the first semester of 2018 and stay there for about two weeks. Prof. Colm will visit the group again in 2019.
\item Samuel Johnson will visit the group at USP in the second semester 2018 for about 10 days.
\item Lorenzo Pellis and Louise Dyson will visit the group in S�o Carlos in the first and second semesters of 2019. Their visits will take about two weeks.
\end{itemize}

\item From USP to Warwick:
\begin{itemize}
\item Prof. Francisco will visit the group in Warwick in the second semester of 2018 and second semester of 2019.
\item Thomas Peron will visit the group in Warwick in the first semester of 2018 and second semester of 2019.
\item Guilherme Arruda will visit the group in Warwick in the second semester of 2018.
\item Paulo Ventura will visit the group in Warwick in the first semester of 2019. The visit of these students will take about two weeks.
\end{itemize}
\end{itemize}

{\color{red}We probably need to mention that there would be two possible schedules of visits depending on whether we get the Leverhulme grant or not.}
\subsection{Expected outputs}
The following concrete and quantifiable outputs are expected to result from this partnership:
\begin{itemize}
\item Scientific publications: collaborative work on the scientific objectives described above will lead to joint publications in high-impact international journals.
\item Workshop: Connaughton and Rodrigues hope to co-organise a 2-3 day international workshop on "Dynamics of spatial networks" to coincide with one of Prof. Rodrigues' UK visits. The topic is proposed in response to a recent call for workshop topics from the EPSRC-funded network on Emergence and Physics Far From Equilibrium, of which Connaughton and MacKay are both members. This would allow us to organise an event of international significance at no additional cost to the grant.
\item Joint training of PhD students: PhD students from Warwick's Complexity Science and MathSys CDTs and from USP will also participate in the research described above and enjoy the opprtunity to make exchange visits between USP and UoW.
\item Engagement with wider UK and Brazilian research communities: on the UK side, Prof. Rodrigues will have the opportunity to make seminar visits to complexity science groups at other UK institutions (Manchester, Imperial, Cambridge, Oxford, Bristol, QMUL)  to be organised with local seminar organisers.
\item Lectures:  if the Leverhulme application is successful, Prof. Rodrigues will teach an advanced graduate level course (the Leverhulme Lectures) at UoW consisting of 8 2-hour lectures aimed at graduate students in the Complexity Science and MathSys CDTs. The topic would be "Theory of Networks". They will be delivered via the Mathematics Taught Course Centre (TCC), a joint venture between the maths departments at Bath, Bristol, Imperial, Oxford and Warwick. It uses video-conferencing facilities to organise a series of postgraduate courses each year that can be taken by graduate students at all 5 TCC partners.
\end{itemize}

\section{Impacts of exchange for host and partner institutions}
In signing the SPRINT partnership agreement, both UoW and USP have recognised the growing strategic importance of closer scientific relationships between the UK and Brazil.
From the perspective of WMI, this partnership has the potential to  contribute deeply to the research activities of the CCS and SBIDER. The specialist value that Prof. Rodrigues will bring is his ability to link the more theoretical complexity science and dynamical systems research at Warwick (MacKay, Johnson, Esenturk) with the more applied research being done in epidemiological modelling (Keeling, Pellis, Dyson) and industrial mathematics (Connaughton). At the UK national level, such expertise is particularly important at this point in time as RCUK and the industrial community are increasingly moving towards the view that the significant investments made in complex systems research over the last ten years, including at Warwick, should now be maturing to produce real world applications.  The choice of ICMC as a SPRINT partner is therefore strongly aligned both with local research priorities at UoW and with wider national research priorities. The potential benefits will be realised through the outputs mentioned above

{\color{red} Need a few lines from Francisco about how this project will benefit ICMC and USP}


