\section{Summary of proposed exchange activities}
A partnership and scientific exchange programme is proposed between the 
University of S\~{a}o Paulo (USP) and the University of Warwick (UoW). The USP 
partner is the Institute of Mathematics and Computer Science (ICMC). The UoW 
partners are the Centre for Complexity Science (CCS) and the Zeeman Institute 
for Systems Biology and Infectious Disease Epidemiology Research (SBIDER), two  
interdisciplinary joint ventures of the Warwick Mathematics Institute (WMI). The USP 
project leader is Prof. Francisco A. Rodrigues, an associate professor and head 
of the Complex Systems group at the ICMC. The UoW project leader is Dr. Colm 
Connaughton, a reader in applied mathematics and director of the CCS. The 
scientific theme is dynamical processes on 
multi-layer and dynamic networks. It relates to the FAPESP grant "Information 
spreading in complex networks" (2016/25682-5) held by Prof. Rodrigues. The high level aims of the exchange are:
\begin{enumerate}
\itemsep\myitemsep
\item to improve scientific understanding of the dynamics of coherent phenomena 
in dynamical and multilayer networks.
\item to forge a new and enduring collaboration on complex systems research between UoW and USP.
\end{enumerate}
A detailed description of the scientific objectives of the collaboration is 
provided in Sec.~\ref{sec-science}. The proposed exchange programme will run 
for two years spanning the 2017-18 and 2018-19 academic years. The first year 
will see visits to UoW by members of the USP team and the second year will see 
return visits to USP by members of the UoW team. A detailed description of the 
proposed schedule is provided in Sec.~\ref{sec-timeline}. In parallel to this 
proposal, Connaughton and Rodrigues have jointly prepared a complementary 
proposal to be submitted to the Leverhulme Trust (deadline 11 May 2017) seeking 
a Leverhulme Visiting Professorship to be held by Prof. Rodrigues at UoW for 
the 2017-18 academic year. If funded, this complementary proposal will allow us 
to significantly enhance both the breadth and depth of the proposed partnership.

\section{Description of research team}

\subsection{Partner institutions}
The {\hvnb Centre for Complexity Science (CCS)} at the UoW was founded in 2007 
to host one of the first EPSRC Centres for Doctoral Training (CDT) on the topic 
of complex systems. It grew rapidly to become one of the largest complexity 
science research groups in the UK. It currently hosts 8 academic staff, several 
postdoctoral researchers and about 50 graduate students. It co-hosts the 
\pounds 3.4M EPSRC-MRC Centre for Doctoral Training in Mathematics of Real 
World Systems in partnership with the {\hvnb Zeeman Institute for 
Systems Biology and Infectious Disease Epidemiology Research (SBIDER)} . SBIDER was 
established in 2016 when the Warwick Systems Biology Centre and Warwick 
Infectious Disease and Epidemiology Research Centre joined forces to form one 
of the largest mathematical biology research groups in the UK. It is the coordinating node of a 
\$10M global consortium funded by the Gates Foundation to develop the modelling 
capability to support the UN 2020 goals for the elimination of Neglected 
Tropical Diseases. Both the CCS and SBIDER are interdisciplinary joint ventures 
of the {\hvnb Warwick Mathematics Institute (WMI)}. WMI ranked 3rd in the UK for mathematical 
sciences in the 2014 Research Excellence Framework (the most recent UK national 
research assessment exercise) and 32nd globally in the 2016 QS world university 
subject rankings. In 2014, Regius Professor Martin Hairer was awarded a fields 
medal. UoW is therefore ideally placed to participate in and benefit from 
international partnerships with distinguished researchers like Prof. Rodrigues. 
USP is an eighty-year old research-led university having the highest QS ranking in Latin America in the 
last two years. Its {\hvnb Institute of  Mathematical and Computer Sciences (ICMC)} was created in 
1971 and is one of the top Brazilian institutions in the fields of computer science, informatics, 
mathematics, applied mathematics and statistics, for both research and training. ICMC has more than 
150 faculty members with an annual peer-reviewed publication output of about 450 papers. The 
Complex Systems group at ICMC was founded in 2014 and currently hosts 5 
academic staff and  more than 30 students and post-docs working on complex networks, nonlinear 
dynamics, control theory and probability.  

\subsection{University of S\~{a}o Paolo Team}
\begin{itemize}
\itemsep\myitemsep
\item {\hvnb Prof. Francisco A. Rodrigues (USP team leader)}  is an associate professor at USP.  He is the head of the Complex Systems group at the ICMC. His background is in theoretical physics. He has published over 60 papers in leading peer-reviewed scientific journals including Physical Review Letters, Physics Reports and PLoS One with more than 2500 citations to date (Google Scholar). His research specialisation is the interaction between structure and dynamics in the theory of complex networks. His recent work has focussed on the development of concepts and methods for characterisation of complex networks, the study of epidemic and rumour spreading in social and technological networks, the modelling of synchronisation of coupled oscillators and the development of new methods for pattern recognition and data mining.  His current and former research students include five PhD students and nine masters students. He has coordinated multiple  successful FAPESP projects, including international projects.

\item {\hvnb Prof. Luciano da Fontoura Costa} is a full professor at the Institute of Physics at S�o Carlos. He holds a degree in Electrical Engineering and a PhD in Physics (Kings College London). He published more than 280 papers in scientific journal, receiving more than 11,000 citations (h-index 45) (Google Scholar). He advised more than 50 graduate students and coordinated several research projects, including international projects like Human Frontiers. He is an expert in complex networks, image analysis, electronics and data mining.

\item {\hvnb Mr. Thomas Kau\^{e} Dal Maso Peron}  is a PhD student advised by Prof. Rodrigues.  He holds a degree in Physics (2010) and a MSc in computational physics. He has already published over 20 scientific articles, including a paper in Physics Reports. He will finish his PhD course in the second semester of 2017. FAPESP has already approved his post-doctoral  fellowship to start this year. He is an expert in synchronization of coupled oscillators.

\item {\hvnb Mr. Guilherme Ferraz de Arruda } holds a degree in Electrical Engineering and MSc in Computer Science. He is finishing the PhD course under the supervision of Prof. Rodrigues. He has published over 10 scientific articles.
\end{itemize}

\subsection{University of Warwick Team}
\begin{itemize}
\itemsep\myitemsep
\item
{\hvnb Dr. Colm Connaughton (UoW team leader)}  is a Reader in the WMI, 
the director of the CCS and the deputy director of the EPSRC-MRC funded Mathematics of Real 
World Systems (MathSys) Centre for Doctoral Training. He is an expert on 
complex systems, fluid dynamics and non-equilibrium statistical mechanics with 
particular expertise on turbulence and the kinetic theory of irreversible 
systems. He has published 51 papers in international peer-reviewed journals and 
is responsible, as PI or CI, for over \pounds 11M of current or previous EPSRC 
CDT support, including the MathSys CDT, plus over \pounds 350K of direct 
research support. 
\item {\hvnb Prof. Matt Keeling:} is a Professor of mathematical epidemiology 
at the University of Warwick with a joint appointment between the Mathematics 
Institute and the School of Life Sciences. His research focuses on the 
development and application of mathematical models of infectious disease 
propagation in human and animal populations.  He has authored over 100 papers 
which have garnered over 11000 citations (Google Scholar). He is currently the 
director of the Zeeman Institute for Systems Biology and Infectious Disease 
Epidemiology (SBIDER).
\item {\hvnb Prof. Robert MacKay FRS FInstP FIMA:} is the Professor of 
Interdisciplinary Mathematics at the Warwick Mathematics Institute, a fellow of 
the Royal Society and former president of the Institute of Mathematics and its 
Applications. He is the former director of the Centre for Complexity Science 
and founding director of the MathSys CDT. He is an expert in dynamical systems, 
mathematical physics and complexity science. He is particularly well known for 
his fundamental contribution to the mathematical theory of synchronisation and 
coherent structures in spatially extended nonlinear systems.   He has authored 
1 book, 136 peer-reviewed journal articles and 50 other articles, and entered 
the ISI Highly cited list under Mathematics in 2008.
\item{\hvnb Dr. Samuel Johnson} is an asst. prof. in the WMI and CCS.  
He is an expert in networks and theoretical ecology.
\item{\hvnb Dr. Louise Dyson} is an asst. prof. in the WMI and SBIDER.  
Her expertise is in social complexity and epidemiology.
\item{\hvnb Dr. Emre Esenturk} is a Marie Curie postdoctoral fellow in the WMI 
and CCS. He is an expert in kinetic theory.
\item{\hvnb Dr. Lorenzo Pellis} is a Henry Dale postdoctoral fellow in the WMI 
and SBIDER. His field of expertise is epidemiology.
\end{itemize}

\section{Scientific objectives}
\label{sec-science}
\subsection{Background and scientific context}
Complex systems are composed of connected elements whose nonlinear interactions give
rise to non-trivial collective properties. Commonly quoted examples include our society, the 
Internet, our brain and cellular interactions. Fundamental
research on complexity therefore spans Mathematics, Physics, Biology,  Computer Science and 
Engineering. Policy and technology applications relevant to this project include the control of 
disease spread in human and animal populations ~\cite{Keeling08} and the control of 
synchronization in data transmission~\cite{Pikovsky03}. In the 1990's it was
recognised that many complex systems have a special hierarchical structure: the number of elements
with a given connectivity decreases with connectivity in an approximately scale-free
way~\cite{Barabasi99}. Networks provide a natural framework for modelling this structure and
are thus fundamental in the study of complex systems. Network representations 
permit investigation of how structural properties
of networks affect dynamical processes in the system and the resulting collective properties~\cite{Barrat08}. To give some examples, the epidemic threshold for disease propagation depends on 
network  heterogeneity and is close to zero  in a scale-free network due to
highly connected agents acting as super-spreaders. Network structure controls the
ability of networks of coupled oscillators, like power grids and brains~\cite{Arenas08} ,
to undergo collective synchronisation. Other examples, include 
cascading failures in engineered systems, percolation in materials and consensus formation in
voter models of opinion dynamics. 

Until very recently, most research considered static networks with
one interaction mode~\cite{Costa011}. Most natural and 
artificial systems however evolve with time and have multiple modes of interaction.  
As people move from home to work their contacts  change. Flocks of birds form 
complex  dynamical patterns across a range of length scales with each bird interacting
with a constantly evolving set of neighbours. Such situations are better modelled as dynamical 
agents moving on a surface,  defining interaction networks that change with time~\cite{Buscarino08, 
Levis017}.  In urban transport systems, people can use and switch between different
transport modes.  Objects known as multilayer networks~\cite{Boccaletti014} can 
represent these multiple channels of connections, with each layer defining a  type of interaction~.  The study of systems with networks of moving agents and multilayer interactions is thus at the cutting edge of research in the field and there are many
open questions to be addressed. This project will answer some of them, building on the combined strengths of the UoW and USP teams. We now list three specific scientific objectives which
we will pursue during the project along with the team members whose expertise is required.

\subsection{Objective 1:  Synchronisation of moving agents (Rodrigues, MacKay, 
Connaughton, Johnson, Esenturk)}
Synchronization occurs when a set of self-sustained oscillators are weakly 
connected as a network. Network structure influences the emergence of the 
synchronous state. In the case of Kuramoto oscillators, the critical coupling 
strength for the emergence of synchronization depends inversely on the second 
moment of the degree distribution. Most studies on synchronization have considered 
static networks. However, recently Levis et al.~\cite{Levis017} have started to  
consider oscillators interacting in a dynamic network generated by mobile agents 
moving in a two-dimensional space and interacting inside a fixed radius. 
Our goal is to generalize  this study to non-flat surfaces. We will implement a 
model of particles moving in a rough relief whose velocities change according 
to their positions. Each agent is a Kuramoto oscillator and the interaction 
between pairs of oscillators occurs according to their distance. This analysis 
will enable to verify how such terrain affects the emergence of the synchronous 
state and understand the robustness of the spatial synchronization mechanism
identified in~\cite{Levis017} . This effort will combine the physical approach to 
synchronization of Prof. Rodrigues with the rigorous mathematical 
approach of Prof. MacKay and the expertise of Connaughton and Esenturk
on kinetic theory of mass transport models.  Dr. Johnson will contribute  his experience on 
complex networks theory and modelling of complex systems.

\subsection{Objective 2: Epidemic spreading in dynamic and multilayer networks (Rodrigues, 
Keeling, Dyson, Pellis)}

The study of information propagation on networks has applications to modelling and forecasting the spread of diseases and rumours.
Prof. Rodrigues' group have been working with epidemic models on networks for five years ~
\cite{Arruda014, Arruda017}. In collaboration with SBIDER researchers Keeling, Dyson and Pellis, 
we will extend the standard susceptible-infected-recovered (SIR) epidemic model to populations of mobile agents and to populations 
whose interactions are represented as multilayer networks. In the former case, 
building on synergy with Objective 1, agents travel in a two-dimensional 
space and interact when their distance is below a given threshold. A
simple variant of such a model has been proposed in~\cite{Buscarino08}.
We will extend the approach to include important practical aspects of epidemic 
dynamics, such as heterogeneous transmission and agents moving with different 
velocities. Our goal is to understand dynamical processes, such as the epidemic threshold. 
In the latter case of multilayer representation, our goal is different. We aim to understand how
spread of awareness about a disease affects its spread. In one layer, 
information about a disease is shared among users via a social network. 
This information affects the transmission of the disease on the other layer of physical 
contacts. We will improve on previous work ~\cite{Funk09, Granell013} by using more appropriate models for information 
transmission, like the one introduced by Moreno et al.~\cite{Moreno04}. 



\subsection{Objective 3: Network methods for analysis of turbulence (Rodrigues, 
Connaughton)}
This is the most speculative of our objectives. Recently, several methods have been 
developed to transform time series into networks ~\cite{Lacasa08, 
Marwan09, Donner010}. These completely novel approaches to time series have been used 
to find patterns of activity in dynamical systems such as identification of transition periods in 
paleoclimate data related to human evolution using recurrence network analysis~\cite{Donges011}.  
We will apply these approaches to Lagrangian timeseries in turbulence. Lagrangian measurements are obtained following the path of a tracer particle in the flow and naturally produce time series. Our hope is
 to identifying the elusive links between patterns in Lagrangian time series and large scale coherent
 structures, an issue which is still poorly understood in the turbulence community. While there is 
 no consensus on what is the "best" network representation for a given time series, it is plausible
 that the ability of network quantities to capture global information about a signal may unveil
 patterns in Lagrangian data that are invisible to time-local analyses. This aspect of the project will
 combine Connaughton's expertise in fluid dynamics with Rodrigues' expertise in network theory
 and pattern recognition.

\subsection{Relationship to ongoing FAPESP grant}

The associated FAPESP grant  "Information spreading in complex 
networks" (2016/25682-5) is held by Prof. Rodrigues at ICMC. It applies 
statistical inference to quantify how network properties 
influence the propagation of information on networks. Details are
provided in the attached summary. Objective 2 directly extends this work to 
spatially mobile populations. Objective 1 is also closely connected since
synchronisation results from the spread of dynamical correlations. The project will benefit 
from the more applied perspective on epidemiology available 
in SBIDER (Keeling, Dyson, Pellis) and from the more theoretical perspective on synchronisation
and statistical physics available in WMI (MacKay, Connaughton, Johnson).

\section{Management, timeline and and expected outputs}
\label{sec-timeline}
\subsection{Management and scientific schedule}
The project will be co-managed by Connaughton and Rodrigues following the 
rules and procedures of UoW and USP. All objectives will be pursued in parallel, 
since there are many cross-links. An estimated scientific schedule is as
follows:

\begin{itemize}
\itemsep\myitemsep
\item {\hvnb Objective 1:} will  start with some simulations of synchronization in a two 
dimensional flat space and build from there. The USP group will implement the software 
with help of UoW researchers. This will take about three 
months. Subsequent theoretical analysis will be done jointly and is expected to take about 12 
months.

\item {\hvnb Objective 2:}  will take about 18 months. Initially, we will adapt the 
models developed by the Warwick group to multilayer and temporal networks. 
Next, we will generalize disease transmission models to temporal networks made 
up of moving agents. Both groups will perform the implementation of the models.

\item {\hvnb Objective 3:} will take about 18 months. 
The USP group will implement and validate several algorithms to 
transform time series into networks and apply them to numerical data generated 
by the UoW group.  Then, we will jointly try to determine whether we can learn
something about turbulent flows.
\end{itemize}

\subsection{Schedule of exchange visits}

\begin{center}

\begin{tikzpicture}[y=0.50cm]  

\begin{ganttchart}[
hgrid,
vgrid,
bar/.append style={fill=YellowGreen},
x unit=12mm,
y unit title=7.5mm,
y unit chart=7.5mm,	
title height=.75, title top shift=0,
title label anchor/.style={below=-1.5ex},
bar top shift=0.2, bar height=0.6,
time slot format=isodate-yearmonth,
compress calendar
]{2017-10}{2018-09}

\gantttitlecalendar{year, month=shortname} \\
\ganttbar{USP to UoW}{2017-10}{2017-12} 
\ganttbar[inline=true]{\textcolor{blue}{CC \& EE}}{2017-10}{2017-12}
\ganttbar[inline=true]{\textcolor{blue}{SJ}}{2018-04}{2018-06}\\
\ganttbar{UoW to USP}{2018-01}{2018-03} 
\ganttbar[inline=true]{\textcolor{blue}{TP}}{2017-10}{2017-12}
\ganttbar[inline=true]{\textcolor{blue}{FR}}{2018-01}{2018-03}

\end{ganttchart}

\end{tikzpicture}
\end{center}
%\vspace{0.5cm}
\begin{center}
\begin{tikzpicture}[y=0.50cm]  

\begin{ganttchart}[
hgrid,
vgrid,
bar/.append style={fill=YellowGreen},
x unit=12mm,
y unit title=7.5mm,
y unit chart=7.5mm,	
title height=.75, title top shift=0,
title label anchor/.style={below=-1.5ex},
bar top shift=0.2, bar height=0.6,
time slot format=isodate-yearmonth,
compress calendar
]{2018-10}{2019-09}
\gantttitlecalendar{year, month=shortname} \\
\ganttbar{USP to UoW}{2018-10}{2018-12}
\ganttbar[inline=true]{\textcolor{blue}{LP or LD}}{2018-10}{2018-12}
\ganttbar[inline=true]{\textcolor{blue}{LP or LD}}{2019-04}{2019-06}
\ganttbar[inline=true]{\textcolor{blue}{CC}}{2019-07}{2019-09}
\\
\ganttbar{UoW to USP}{2019-01}{2019-03}
\ganttbar[inline=true]{\textcolor{blue}{FR, TP \& LC}}{2019-01}{2019-03}
\ganttbar[inline=true]{\textcolor{blue}{GA}}{2019-04}{2019-06}
\end{ganttchart}

\end{tikzpicture}

\end{center}

\noindent In this project schedule, researchers are identified by their initials and bars indicate windows to schedule visits. Flexibility is required given academics' busy diaries.  The budget is for visits of two weeks.  If the Leverhulme bid succeeds, this schedule will be revised in consultation with FAPESP since Rodrigues would be funded to spend the  2017-18 year at UoW.


%To develop the research proposed in this project, several meetings are 
%necessary to discuss the results and further research steps. Thus, basically, 
%we plan to have the following visits:
%\begin{itemize}
%\itemsep\myitemsep
%\item From Warwick to USP:
%\begin{itemize}
%\item Prof. Colm  and Emre Esenturk will visit the group in S\~{a}o Carlos in 
%the first semester of 2018 and stay there for about two weeks. Prof. Colm will 
%visit the group again in 2019.
%\item Samuel Johnson will visit the group at USP in the second semester 2018 
%for about 10 days.
%\item Lorenzo Pellis and Louise Dyson will visit the group in S\~{a}o Carlos in 
%the first and second semesters of 2019. Their visits will take about two weeks.
%\end{itemize}
%
%\item From USP to Warwick:
%\begin{itemize}
%\itemsep\myitemsep
%\item Prof. Francisco will visit the group in Warwick in the second semester of 
%2018 and second semester of 2019.
%\item Thomas Peron will visit the group in Warwick in the first semester of 
%2018 and second semester of 2019.
%\item Guilherme Arruda will visit the group in Warwick in the second semester 
%of 2018.
%\item Paulo Ventura will visit the group in Warwick in the first semester of 
%2019. The visit of these students will take about two weeks.
%\end{itemize}
%\end{itemize}
%
%{\color{red}We probably need to mention that there would be two possible 
%schedules of visits depending on whether we get the Leverhulme grant or not.}


\subsection{Expected outputs}
The following concrete and quantifiable outputs are expected to result from 
this partnership:
\begin{itemize}
\itemsep\myitemsep
\item Scientific publications: collaborative work on the scientific objectives 
described above will lead to joint publications in high-impact international 
journals.
\item Workshop: Connaughton and Rodrigues hope to co-organise a 2-3 day 
international workshop on "Dynamics of spatial networks" during 
Prof. Rodrigues' second visit. This was proposed in response to a recent 
call for workshop proposals from the EPSRC-funded network on Emergence and Physics 
Far From Equilibrium, of which Connaughton and MacKay are members. This 
would be an event of international significance at no 
additional cost to the grant.
\item Training of PhD students: PhD students from UoW's Complexity 
Science and MathSys CDTs and from USP will also participate in the research 
described above and enjoy the opportunity for exchange visits between USP 
and UoW.
\item Engagement with wider UK and Brazilian research communities: on the UK 
side, Prof. Rodrigues will have the opportunity to make seminar visits to 
complexity science groups at other UK institutions (Manchester, Imperial, 
Cambridge, Oxford, Bristol, QMUL)  to be organised with local seminar 
organisers.
\item Lectures:  if the Leverhulme application is successful, Prof. Rodrigues 
will teach an advanced graduate level course (the Leverhulme Lectures) at UoW 
consisting of 8 2-hour lectures aimed at graduate students in the Complexity 
Science and MathSys CDTs. The topic would be "Theory of Networks". They will be 
delivered via the Mathematics Taught Course Centre (TCC), a joint venture 
between the maths departments at Bath, Bristol, Imperial, Oxford and Warwick. 
It uses video-conferencing facilities to organise a series of postgraduate 
courses each year that can be taken by graduate students at all 5 TCC partners.
\end{itemize}

\section{Impacts of exchange for host and partner institutions}
In signing the SPRINT partnership agreement, both UoW and USP have recognised 
the growing strategic importance of closer scientific relationships between the 
UK and Brazil.
From the perspective of WMI, this partnership has the potential to  contribute 
deeply to the research activities of the CCS and SBIDER. The specialist value 
that Prof. Rodrigues will bring is his ability to link the more theoretical 
complexity science and dynamical systems research at Warwick (MacKay, Johnson, 
Esenturk) with the more applied research being done in epidemiological 
modelling (Keeling, Pellis, Dyson) and industrial mathematics (Connaughton). At 
the UK national level, such expertise is particularly important at this point 
in time as RCUK and the industrial community are increasingly moving towards 
the view that the significant investments made in complex systems research over 
the last ten years, including at Warwick, should now be maturing to produce 
real world applications.  The choice of ICMC as a SPRINT partner is therefore 
strongly aligned both with local research priorities at UoW and with wider 
national research priorities. The potential benefits will be realised through 
the outputs mentioned above

From the ICMC side, the collaboration with UoW will reinforce some core ICMC research areas. Researchers at ICMC have experience with theoretical epidemiology, but will benefit greatly from the closeness of the UoW group of Prof. Keeling to real world data and applications. The application of network methods to
turbulence data is a novel idea which has the potential to produce something completely new
for both groups. UoW also has very complementary theoretical expertise in dynamical systems (MacKay) which will be of value to the ICMC team, Finally, contact with the multi-disciplinary scientific environment at the CCS will greatly improve the scientific experience of younger USP team members.



